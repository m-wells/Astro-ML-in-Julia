% vim:ft=tex:
%
\documentclass[12pt]{article}

\title{
	Clustering of Cross-Referenced Astronomical Data Sets
}
\author{
	Vera Abaimova --- \texttt{stormseecker@gmail.com} \\ 
        Mark Wells --- \texttt{mwellsa@gmail.com}
}

\date{December 9, 2014}
\begin{document}
\maketitle

\begin{abstract}
The large amount of astronomical data that is now available from a multitude of missions creates a need for machine learning methods that can analyze it and glean information about our universe.
One challenge in particular is to analyze data about the same observations made by different missions, \textit{i.e.}, analyze cross-referenced data, especially data in different formats.
One such format is time series data provided by the \textit{Kepler} mission, which results in additional difficulties.
Our proposed method extracts features from cross-referenced data sets, including time series data, and applies a hierarchical clustering model in order to aid stellar classification.

\end{abstract}

\section{Introduction} % (fold)
\label{sec:Introduction}

% section Introduction (end)

\section{Related Work} % (fold)
\label{sec:Related Work}

% section Related Work (end)

\section{Problem Definition} % (fold)
\label{sec:Problem Definition}

% Some of the below values are now different thanks to our additional work

The input of our clustering problem is a list of $x_i,_j$ where $x_i$ is the star object, numbered from 1 to 20,840 and $j$ is the feature, numbered from 1 to 29.
Our feature space includes features such as $fuv\_mag$, $nuv\_flux$, and $kepmag$, among others, where $fuv\_mag$ is the far ultraviolet magnitude, $nuv\_flux$ is the near ultraviolet, and $kepmag$ is a measure of brightness of an object in the \textit{Kepler} pass band.

Our output is a list of $x_i,_j \in C_k$ where $C$ is cluster and $k$ denotes cluster number. 

% section Problem Definition (end)

\section{Method} % (fold)
\label{sec:Method}

% section Method (end)

\section{Experiments} % (fold)
\label{sec:Experiments}

% section Experiments (end)

\section{Summary} % (fold)
\label{sec:Summary}

% section Summary (end)

\end{document}
