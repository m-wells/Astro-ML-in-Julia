\subsection{Preprocessing}
\label{sub:Preprocessing}
Our data from the \textit{Kepler} spacecraft is stored in \textit{FITS} files (Flexible Image Transport System).
These files can contain \emph{headers} and \emph{tables}.
We also obtained data from the \textit{Galex} mission.
All of this data was processed using \emph{Julia} (\url{http://julialang.org/}).
In order to expedite this process we broke up the dataset and distributed it over several systems.

\subsubsection{Header Information}
\label{ssub:Header Information}
For our data set the headers will contain some physical properties that we will be interested in using and the tables will contain the time series data.
These keywords, shown in Table~\ref{tab:keywords}, are provided by the \textit{Kepler} team.
\begin{table}
    \begin{center}
    \begin{tabular}{|c|}
        \hline
        KEPLERID\\
        GMAG\\
        RMAG\\
        IMAG\\
        ZMAG\\
        D51MAG\\
        JMAG\\
        HMAG\\
        KMAG\\
        KEPMAG\\
        GRCOLOR\\
        JKCOLOR\\
        GKCOLOR\\
        TEFF\\
        LOGG\\
        FEH\\
        EBMINUSV\\
        AV\\
        RADIUS\\
        \hline
    \end{tabular}
    \end{center}
    \caption{A list of the keywords that were extracted from the headers of the FITS files.}
    \label{tab:keywords}
\end{table}

\subsubsection{Time Series}
\label{ssub:Time Series}
A single FITS file will contain the lightcurve of one quarter.
A quarter is one season, i.e. three months.
This is because the spacecraft will have to rotate itself to keep its solar panels directed toward the sun.
This means that for one star we will can have as many as 16~separate files that will need to be stitched together.
The time series data consists of two separate vectors: time and flux.
The flux mean level will be different for each quarter.
